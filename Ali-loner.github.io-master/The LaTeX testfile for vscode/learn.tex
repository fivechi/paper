\documentclass[a4paper]{report}
\usepackage[margin=1in]{geometry}
\usepackage{ctex}
\usepackage{listings}
\usepackage{xcolor}
\usepackage{hyperref}
% 添加关联目录的链接与设置
\hypersetup{
    colorlinks=true,
    linkcolor=black,
    filecolor=magenta,
    urlcolor=cyan
}
% 中英文摘要
\newcommand{\enabstractname}{Abstract}
\newcommand{\cnabstractname}{摘要}
\newenvironment{enabstract}{%
  \par\Large
  \noindent\mbox{}\hfill{\bfseries \enabstractname}\hfill\mbox{}\par
  \vskip 2.5ex}{\par\vskip 2.5ex}
\newenvironment{cnabstract}{%
  \par\Large
  \noindent\mbox{}\hfill{\bfseries \cnabstractname}\hfill\mbox{}\par
  \vskip 2.5ex}{\par\vskip 2.5ex}
% python代码显示设置
\definecolor{codegray}{gray}{0.98}
\definecolor{codepurple}{rgb}{0.58,0,0.82}
\lstdefinestyle{mystyle}{
    backgroundcolor=\color{codegray},   
    commentstyle=\color{gray},
    keywordstyle=\color{blue},
    numberstyle=\tiny\color{gray},
    stringstyle=\color{codepurple},
    basicstyle=\ttfamily\small,
    breaklines=true,                 
    captionpos=b,                    
    keepspaces=true,                 
    numbers=left,                    
    numbersep=5pt,                  
    showspaces=false,                
    showstringspaces=false,
    showtabs=false,                  
    tabsize=4
}
\lstset{style=mystyle, language=Python}
%设置目录显示的深度为section
\setcounter{tocdepth}{1} 
% title page
\title{基于布朗粒子运动的微小位移测量实验设计与验证}
\author{E}
\date{\today}
% 正文部分
\begin{document}
\maketitle

% 摘要部分(中文)
\begin{cnabstract}
\large
摘要正文:  
物质的性质与其内部结构密切相关,因此在物理学、化学、生物学、材料学、医学等方
面,对物质内部结构的研究是必不可少的。最常见的一种方法是利用光子所产生的物质衍射
干涉图样来研究其内部结构。然而实验得到合适的衍射干涉图样非常复杂,如何高效模拟产
生复杂结构的干涉和衍射对物质内部结构的研究具有重要意义。本实验从两个角度研究复杂
结构的干涉和衍射:一、利用结构光场,这里的结构光场主要是指一种全新的光子自由度-
光子轨道角动量(OAM)光束,也称为涡旋光束及其叠加态形成的具有规则空间光场分布;
二、利用分形复杂孔模拟物质内部的基本复杂结构。通过利用光场调控领域中常用的光学设
备-空间光调制器高效便捷的实现对上述两种情况的实验研究及相应的实验结果分析处理。
\par\textbf{关键字: } 关键字1,关键字2,关键字3
%“\par在段首,表示另起一行,“\textbf{}”,花括号内的内容加粗显示
\end{cnabstract}
% 摘要部分(英文)
\newpage
\begin{enabstract}
\large
English abstract:  
The properties of matter are closely related to its internal structure, so the study of the internal 
structure of matter is essential in physics, chemistry, biology, materials science, medicine and other 
fields. The most common method is to use the diffraction interference pattern produced by photons 
to study the internal structure of matter. However, it is very complicated to obtain suitable 
diffraction interference patterns experimentally. How to efficiently simulate the interference and 
diffraction of complex structures is of great significance for the study of the internal structure of 
matter. This experiment studies the interference and diffraction of complex structures from two 
aspects: one, using structured light fields, where the structured light fields mainly refer to a new 
degree of freedom of photons - orbital angular momentum (OAM) beams, also known as vortex 
beams and their superposition states forming regular spatial light field distributions; two, using 
fractal complex apertures to simulate the basic complex structures inside matter. By using a 
common optical device in the field of light field manipulation - spatial light modulator, the 
experimental study and corresponding experimental results analysis and processing of the above 
two situations are efficiently and conveniently realized.
\par\textbf{Keywords:} keyword1, keyword2, keyword3
%“\par在段首,表示另起一行,“\textbf{}”,花括号内的内容加粗显示
\end{enabstract}


\tableofcontents
\chapter{题意解析}

\section{题目:基于布朗粒子运动的微小位移测量实验设计与验证}
这里是section部分
\subsection{目的}
这里是subsection部分

\section{目标定位}

\chapter{实验原理}

\chapter{附录}
\section{代码展示}
\subsection{网络模型}
\begin{lstlisting}[language=Python, caption=生成器, label=code:generator]

\end{lstlisting}
\subsection{图片预处理}
\begin{lstlisting}[language=Python, caption=图片处理, label=code:discriminator]

\end{lstlisting}
\subsection{训练}
\begin{lstlisting}[language=Python, caption=训练, label=code:discriminator]

\end{lstlisting}
\end{document}